\documentclass[12pt]{report}
\begin{document}
\center{\bf MATHEMATICS 5101}\\ \center{\bf Mathematical Principles in
  Science I: Linear Mathematics in Finite Dimensions}
\begin{itemize}
\item[Time:]	MWF 9:10-10:05am
\item[Prerequisites:] Basics of matrix algebra (as e.g. in Math 2568 or 2174); mathematical maturity
\item[Texts:] 1.\emph{Introduction to Linear Algebra} by L.W. Johnson,
R.D. Riess,and J.T. Arnold
%\item[] 

2.\emph{Linear Algebra and Its Applications} by G. Strang
\item[Web site:]
http://www.math.ohio-state.edu/$\sim$gerlach/math
\item[Syllabus (in essence):]
Vector spaces\\
Space of duals (= covectors = linear functionals)\\
Inner products\\
Linear operators\\
Eigenvalues \& eigenvectors\\
Adjoint of an operator\\
Quadratic forms\\
Small oscillations\\
Singular value decomposition
\item[Homeworks:] One homework set every week, generally handed out
each Friday and due the following Friday AT THE START OF CLASS.
\item[Midterm Exam:] TBA, whether and when; if so, to be announced two weeks in advance. 
\item[Final Exam:]	One take-home comprehensive final
\item[Who is Who \& Office:]
Ulrich Gerlach\\
MW124 or MW506 in Mathematics Tower\\
Telephone number: 292-2560 or 292-7235\\
e-mail:  gerlach.1@osu.edu
\item[Office hours:]	MWF 2:00-3:00pm or by appointment.%Usually after class
\item[Homework policy:]For each assignment, TEAMS are allowed and even
encouraged, with a limit of 3 persons per team.  Each team
submits ONE SET OF SOLUTIONS, signed by each member, and
every team member receives the same grade.\\
Teams disband after each assignment.  Teams then re-constitute
for every assignment.  This way you are encouraged to select
partners who contribute to the final product that you hand in.
\item[Grading Guidelines:] Each assignment paper will be graded for
mathematical correctness AND PRESENTATION.  
  \begin{itemize}
    \item
      Points will be deducted for sloppiness, incoherent or insufficient explanation, or for lack of supporting rationale.  
    \item 
      The solutions should be presented so that your fellow students AND YOUR PROSPECTIVE CLIENT could read them and follow both the calculations and logic.\\
\end{itemize}
Each assignment (8 or 9 total) will consist of approximately 100 
possible points, and the Final Exam will be worth about 200 points.  
There is a total of about 1000 points. 
% In addition, your Final Exam
%percentage may replace up to 2 weekly assignment grades if that
%benefits you. 
Late papers will not be accepted except in extreme
situations with documented excuse. 
%the first two unexcused missed 
%assignments will be replaced by the Final Exam percentage, with 
%subsequent events assigned a grade of zero.  
It is the student's
responsibility to be aware of all instructions that are delivered
during class, including departures from general assignments.
\item[Use of software:]
You are encouraged and sometimes obligated to use a 
software package such as Maple or Matlab.  So, practice with 
some linear algebra software soon, and get used to working with
it.  All routine calculations should be checked this way.  If
we want you to do hand calculations, we will make it explicit.
Even then, check yourself.  Moreover, when you use software, you
must acknowledge that you did, and support the output with some
form of explanation:  why it was used, and an interpretation of
any answer that is not just a routine calculation.  A simple
solution consisting of output from Matlab or Maple is not sufficient.
Common sense should rule here.
\end{itemize}
\end{document}
